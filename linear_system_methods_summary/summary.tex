\documentclass[9pt, serif]{beamer}

\newlength{\wideitemsep}
\setlength{\wideitemsep}{\itemsep}
%\addtolength{\wideitemsep}{9pt}
\let\olditem\item
\renewcommand{\item}{\setlength{\itemsep}{\wideitemsep}\olditem}

% This file is a solution template for:

% - Talk at a conference/colloquium.
% - Talk length is about 20min.
% - Style is ornate.

\usepackage[noend]{algpseudocode}
\usepackage{graphics}
\usepackage{marvosym}
\usepackage{latexsym}
\usepackage{amsfonts}
\usepackage{amsmath}  
\usepackage{epstopdf}
\usepackage[english]{babel}
\usepackage[latin1]{inputenc}
\usepackage{helvet}
\usepackage[T1]{fontenc}


\newcommand{\bi}{\begin{itemize}}
\newcommand{\be}{\begin{enumerate}}
\newcommand{\ei}{\end{itemize}}
\newcommand{\ee}{\end{enumerate}}
\newcommand*\Let[2]{\State #1 $\gets$ #2}
\newcommand{\abs}[1]{\lvert#1\rvert}
\newcommand{\norm}[1]{\lVert#1\rVert}


\mode<presentation>
{
% THEME CHOICES
%\usetheme{CambridgeUS}
\usetheme{Warsaw}
%\usetheme{Rochester}
%\usetheme{Bergen}
%\usetheme{Berlin}
%\usetheme{Copenhagen}
%\usetheme{Boadilla}
%\usetheme{PaloAlto}

% SET BACKGROUND COLORS
\setbeamercolor{background canvas}{bg=white}
\setbeamerfont{frametitle}{size=\large}
}
\setbeamercolor{frame}{bg=mygold}


\title[Methods of Solution for Linear Systems]
{A Summary of Methods of Solution for Linear Systems}
\author[]
{Nate DeMaagd, Kurt O'Hearn}
\institute[Grand Valley State University]
{MTH 499-02}
\date{April 23, 2013}


\begin{document}

\begin{frame}
    \titlepage
\end{frame}


\begin{frame}{Outline}
    \pause
    \bi
        \item Types of methods \pause
        \item Direct methods and examples \pause
        \item Indirect methods and examples
    \ei
\end{frame}


%INTRODUCTION
\begin{frame}{Overview of Types of Methods for Solving Linear Systems}
    \pause
    Recall: For the linear system $Ax = b$ with $A_{m\times n}$: \\
    \pause
    \begin{center}
        \begin{tabular}{l|l}
            System Type & Possible Number of Solutions \\ \hline
            Square ($m = n$) & None, Unique \\
            Overdetermined ($m > n$) & None, Unique \\
            Underdetermined ($m < n$) & None, Infinite \\
        \end{tabular}
    \end{center}
    \pause
    Types of Methods
    \bi
        \item Direct methods \pause
        \bi
            \item Execute a predetermined number of computations to produce a result \pause
            \item Methods: compute $A^{-1}$, transform $A$ using factorizations/pivoting \pause
        \ei
        \item Indirect methods \pause
        \bi
            \item Generate a sequence of intermediate results which (hopefully) produce the desired final result \pause
	        \item Methods: \pause
            \bi
                \item General: Richardson, Jacobi, Gauss-Seidel, SOR \pause
                \item Symmetric, positive definite: steepest descent, conjugate gradient \pause
            \ei
    	\ei
    \ei
    Note: methods can be general or exploit certain matrix characteristics
\end{frame}


%BASIC CONCEPTS
\begin{frame}{Theory On Iterative Methods for Solving Linear Systems}
    \pause
    Developing A Simple Iterative Method \pause
    \bi
        \item Want to solve: $Ax=b$ for $x$ with $A \in \mathbb{R}^{n \times n}$, $A$ invertible \pause
        \item Introduce an invertible ``splitting matrix'' $Q$ and rearrange to get
	    \begin{align*}
		    Qx = (Q - A)x + b
	    \end{align*} \pause
        \vspace{-5mm}
        \item Now define our iterative process as
	    \begin{align*}
		    Qx^{(k)} = (Q - A)x^{(k-1)} + b
	    \end{align*}
        where $k > 1$ denotes the $k^{\text{th}}$ step in the process \pause
        \bi
            \item Want: each successive iteration to produce a better approximation for $x$ (i.e., converge) \pause
	        \item Also want: algorithms which guarantee convergence after satisfying some conditions \pause
            \item To achieve these ends, we seek $Q$ such that:
            \bi
                \item $\norm{x-x^{(k)}} \to 0$ rapidly, and
                \item $\left[x^{(k)}\right]$ is easy to compute
            \ei \pause
            \item Note: often the initial vector $x^{(0)}$ is an estimate of the solution or arbitrary ($x = 0$)
        \ei
    \ei
\end{frame}


%THEOREM 1
\begin{frame}{Theory On Iterative Methods for Solving Linear Systems Cont.}
    \pause
    Why This Iterative Process Works: \pause
    \bi
        \item Recall iterative process $Qx^{(k)} = (Q - A)x^{(k-1)} + b$ \pause
        \item Letting $k\rightarrow\infty$, solution is $Qx = (Q - A)x + b$ \pause
        \item Assume $Q$ and $A$ nonsingular. So, $Q^{-1}$ and $A^{-1}$ exist and
	    \begin{align*}
	        x^{(k)} = (I - Q^{-1}A)x^{(k-1)} + Q^{-1}b
	    \end{align*}
        \vspace{-3mm}
        \pause
        \item Taking limit of this gives solution
	    \begin{align*}
	        x = (I-Q^{-1}A)x+Q^{-1}b
	    \end{align*}
        \vspace{-3mm}
        \pause

%\item Thus $x \mapsto (I - Q^{-1}A)x + Q^{-1}b$

        \item Thus $x^{(k)} - x = (I-Q^{-1}A)(x^{(k-1)}-x)$ \pause
        \item Select vector norm and subordinate norm so that by using the norm and the recursive definition
	    \begin{align*}
	        ||x^{(k)} - x|| \leq || I -Q^{-1}A||^k~||x^{(0)}-x||
	    \end{align*}
        \vspace{-3mm}
        \pause
        \item Thus, if $|| I - Q^{-1}A|| < 1$, then $\lim\limits_{k\rightarrow\infty} ||x^{(k)} - x || = 0$ 
    \ei
\end{frame}


\begin{frame}{Theory on Iterative Methods for Solving Linear Systems Cont.}
    More General Conditions for Iterative Method Convergence \pause
    \begin{theorem}
        The spectral radius of a matrix $A$, $\rho(A) = \max_{1\le i\le n}\abs{\lambda_i}$, satisfies $$\rho(A) = \inf_{\norm{\cdot}}\norm{A}.$$
    \end{theorem} 
    \pause
    \begin{theorem}
        For the linear system $Ax = b$ with $A$ invertible, define the iteration formula $$x^{(k)} = Gx^{(k-1)} + c.$$
        The sequence $\left[x^{(k)}\right]$ will converge to $(I - G)^{-1}c$ provided that $\rho(G) < 1$.
    \end{theorem}
    \pause
    \begin{corollary}
        The iteration formumla $$Qx^{(k)} = (Q-A)x^{(k-1)}+b$$ will produce a convergent sequence provided that $\rho(I-Q^{-1}A) < 1$.
    \end{corollary} 
\end{frame}


\begin{frame}{Iterative Methods}
    \pause
    \begin{center}
        \begin{tabular}{l|c|c}
            Method & $Q$ & Iteration Formula: $x^{(k)} = (I-Q^{-1}A)x^{(k-1)}+Q^{-1}b$ \\ \hline \hline
            Richardson & $I$ & $x^{(k)} = (I-A)x^{(k-1)}+b = x^{(k-1)}+r^{(k-1)}$ \\
            Jacobi & $D$ & $x^{(k)} = (I-D^{-1}A)x^{(k-1)}+D^{-1}b$ \\
            Gauss-Seidel & $L$ & $x^{(k)} = (I-L^{-1}A)x^{(k-1)}+L^{-1}b$
        \end{tabular}
        \\[5mm]
    \end{center} 
    where
    \bi
        \item $D$: diagonal matrix where $d_{ii} = a_{ii}$
        \item $L$: lower triangular matrix where $l_{ij} = a_{ij}, i \ge j$
    \ei
\end{frame}

\end{document}
