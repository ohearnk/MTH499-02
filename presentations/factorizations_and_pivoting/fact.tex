\documentclass[9pt, serif]{beamer}

\newlength{\wideitemsep}
\setlength{\wideitemsep}{\itemsep}
%\addtolength{\wideitemsep}{9pt}
\let\olditem\item
\renewcommand{\item}{\setlength{\itemsep}{\wideitemsep}\olditem}

% This file is a solution template for:

% - Talk at a conference/colloquium.
% - Talk length is about 20min.
% - Style is ornate.

\usepackage[noend]{algpseudocode}
\usepackage{psfrag}
\usepackage{graphics}
\usepackage{marvosym}
\usepackage{latexsym}
\usepackage{amsfonts}
\usepackage{amsmath}  
\usepackage{bm}
\usepackage{epstopdf}
\usepackage{tikz}
\usetikzlibrary{arrows,matrix,positioning}

\newcommand{\bi}{\begin{itemize}}
\newcommand{\be}{\begin{enumerate}}
\newcommand{\ei}{\end{itemize}}
\newcommand{\ee}{\end{enumerate}}
\newcommand*\Let[2]{\State #1 $\gets$ #2}
\newcommand{\abs}[1]{\left|#1\right|}

\newcommand{\bfr}[1]{\begin{frame}{\hspace{50mm}#1}}
\newcommand{\efr}{\end{frame}}

\definecolor{royalblue}{rgb}{.2,.6,1}%
\definecolor{mygrey}{rgb}{.64,.64,.64}%
\definecolor{mywhite}{rgb}{1,1,1}%
\definecolor{mypurple}{rgb}{.8,.6,1}%

\definecolor{red}{rgb}{1,0,0}
\definecolor{skyblue}{rgb}{0.1960, 0.6000, 0.8000}


\mode<presentation>
{
% THEME CHOICES
%\usetheme{CambridgeUS}
\usetheme{Warsaw}
%\usetheme{Rochester}
%\usetheme{Bergen}
%\usetheme{Berlin}
%\usetheme{Copenhagen}
%\usetheme{Boadilla}
%\usetheme{PaloAlto}

% SET BACKGROUND COLORS

\setbeamercolor{background canvas}{bg=white}
\setbeamerfont{frametitle}{size=\large}
}

\setbeamercolor{frame}{bg=mygold}

\usepackage[english]{babel}

\usepackage[latin1]{inputenc}

\usepackage{helvet}
\usepackage[T1]{fontenc}
\title[Factorizations and Pivoting]
{The Cholesky Factorization and \\Pivoting Methods in Gaussian Elimination}



\author[]
{Nate DeMaagd, Kurt O'Hearn}


\institute[Grand Valley State University]
{MTH 499-02}
  
\date{March 11, 2013}


% If you have a file called "university-logo-filename.xxx", where xxx
% is a graphic format that can be processed by latex or pdflatex,
% resp., then you can add a logo as follows:

\pgfdeclareimage[height=1cm, width=1cm]{GVlogo}{GVSULOGOBLACK}
 \logo{\pgfuseimage{GVlogo}}

% to mask background in logo image
% \pgfdeclaremask{logobacking}{image name }
%


% Delete this, if you do not want the table of contents to pop up at
% the beginning of each subsection:
%\AtBeginSubsection[]
%{
 % \begin{frame}<beamer>{Outline}
 %   \tableofcontents[currentsection,currentsubsection]
 % \end{frame}
%}


% If you wish to uncover everything in a step-wise fashion, uncomment
% the following command:

%\beamerdefaultoverlayspecification{<+->}

\begin{document}

\begin{frame}
    \titlepage
\end{frame}

\begin{frame}{Outline}
    \bi
        \item Cholesky factorization: overview, algorithm, example \pause
        \item Pivoting methods in Gaussian Elimination: connections to factorizations, types, example
    \ei
\end{frame}

%CHOLESKY FACTORIZATION OVERVIEW
\begin{frame}{The Cholesky Factorization}
The Cholesky Factorization \pause
\bi
\item Given $A$, we can write $A$ as  $A=LL^T$ with $L$ lower triangular\pause
\item Necessary conditions on $A$:\pause
	\bi \item Real-valued entries $(A \in \mathbb{R}^{n\text{x}n})$
	\item Symmetric $(A = A^T)$
	\item Positive definite $(x^TAx > 0$ for all $x \in \mathbb{R}^n$, $x \ne 0_n)$ \pause
	\ei
\item Cholesky: special case of $LU$-factorization\pause
\item Why are factorizations important?  Connections later in pivoting \pause
\ei
\end{frame}


%CHOLESKY THM PF
\begin{frame}{Cholesky: Proof Overview}
\begin{theorem}
	If $A$ is a real, symmetric, and positive definite matrix, then $A=LL^T$ for a unique upper triangular $L$. \pause
\end{theorem}
\begin{proof}
    Existence: \pause
    \bi 
    	\item $A$ positive definite $\Rightarrow$ $A$ nonsingular \pause 
		\item Leading principle minors of $A$ positive definite \\$\Rightarrow$ principle minors nonsingular $\Rightarrow$ $A=LU$ ($LU$ factorization exists) \pause
		\item $LU = A = A^T = U^TL^T \Leftrightarrow U(L^T)^{-1} = L^{-1}U^T$ \pause
		\item $U(L^T)^{-1}$ upper triangular, $L^{-1}U^T$ lower triangular \\$\Rightarrow$ both sides diagonal $(D)$ \pause
        \item $U=DL^T$ \pause
	    \item $D$ has positive diagonal entries $\Rightarrow$ $D = D^{1/2}D^{1/2}$ \pause
		\item $A = L^*(L^*)^T$, where $L^* = LD^{1/2}$
    \ei \pause
    Unicity: straightforward (assume two, show same)
\end{proof}
\end{frame}


\begin{frame}{Cholesky Algorithm}
 \begin{algorithmic}[1]
        \Function{Cholesky}{$n, (a_{ij})$}
            \For{$k = 1 \textbf{ to } n$}
                \Let{$l_{kk}$}{$\left(a_{kk}-\sum\limits_{s=1}^{k-1}l_{ks}^2\right)^{1/2}$}
                \For{$i = k + 1 \textbf{ to } n$}
                    \Let{$l_{ik}$}{$\left(a_{ik}-\sum\limits_{s=1}^{k-1}l_{is}l_{ks}\right)/l_{kk}$}
                \EndFor
            \EndFor
            \vspace{-5mm}
            \State \Return{$(l_{ij})$}
        \EndFunction
    \end{algorithmic}
    \vspace{2mm}
    where
    \vspace{2mm}
    \bi
        \item $A = (a_{ij})$: factored matrix
        \item $n$: order of $A$
        \item $L = (l_{ij})$: resulting lower triangular matrix
    \ei
\end{frame}

%CHOLESKY EXAMPLE SLIDE 1
\begin{frame}{Example of Cholesky Algorithm}
    
\begin{columns}
	\column{.35\textwidth} 
	$A=\begin{bmatrix}
	25 & 15 & $-5$ \\
	15 & 18 & 0 \\
	$-5$ & 0 & 11
	\end{bmatrix}$

\vspace{8mm}

	$L=\begin{bmatrix}
 	\visible<2->{5}& 0 & 0\\
 	\visible<3->{3}& \visible<5->{3}& 0\\
	 \visible<4->{$-1$} & \visible<6->{1} & \visible<7->{3} \end{bmatrix}$

\column{.65\textwidth}	
 \begin{algorithmic}[1]
        \Function{Cholesky}{$n, (a_{ij})$}
            \For{$k = 1 \textbf{ to } n$}
                \Let{$l_{kk}$}{$\left(a_{kk}-\sum\limits_{s=1}^{k-1}l_{ks}^2\right)^{1/2}$}
                \For{$i = k + 1 \textbf{ to } n$}
                    \Let{$l_{ik}$}{$\left(a_{ik}-\sum\limits_{s=1}^{k-1}l_{is}l_{ks}\right)/l_{kk}$}
                \EndFor
            \EndFor
            \vspace{-5mm}
            \State \Return{$(l_{ij})$}
        \EndFunction
    \end{algorithmic}
\end{columns}

\vspace{8mm}

\begin{center}
\only<2-2>{$k=1:$\\[3mm] $l_{11} = (25 - 0)^{1/2} = 5$}
\only<3-3>{$k=1,\quad i=2:$\\[3mm] $l_{21} = (15 - 0)/5 = 3$}
\only<4-4>{$k=1,\quad i=3:$\\[3mm] $l_{31} = (-5-0)/5 = -1$}
\only<5-5>{$k=2:$\\[3mm] $l_{22} = (18-3^2)^{1/2} = 3$}
\only<6-6>{$k=2,\quad i=3:$\\[3mm] $l_{32} = \left(0-(-1)(3)\right)/3 = 1$}
\only<7-7>{$k=3:$\\[3mm] $l_{33} = \left(11-(-1)^2-(-1)^2\right) = 3$}
\end{center}
\end{frame}


%CHOLESKY EXAMPLE SLIDE 2
\begin{frame}{Example of Cholesky Algorithm Cont.}
\bi
\item Verify correctness:
\ei
$$L = \begin{bmatrix} 5 & 0 & 0 \\ 3 & 3 & 0 \\ $-1$ & 1 & 3 \end{bmatrix}
\pause\Longrightarrow
L^T = \begin{bmatrix} 5 & 3 & $-1$ \\ 0 & 3 & 1 \\ 0 & 0 & 3 \end{bmatrix}$$
\pause
\vspace{5mm}
$$LL^T = \begin{bmatrix} 5 & 0 & 0 \\ 3 & 3 & 0 \\ $-1$ & 1 & 3 \end{bmatrix}\begin{bmatrix} 5 & 3 & $-1$ \\ 0 & 3 & 1 \\ 0 & 0 & 3 \end{bmatrix}
=\begin{bmatrix}
25 & 15 & $-5$ \\
15 & 18 & 0 \\
$-5$ & 0 & 11
\end{bmatrix} = A$$
\end{frame}


\begin{frame}{Gaussian Elimination: Overview}
    Gaussian Elimination \pause
    \bi
        \item Process which attempts to solve a linear system by reducing the coefficient matrix to triangular form using elementary row operations (the result can then by easily solved using forward/backward substitution) \pause
        \item Possible row operations:
        \bi
            \item Interchange
            \item Scaling
            \item Addition
        \ei \pause
        \item Gaussian elimination only utilizes scaling and addition!
    \ei
\end{frame}


\begin{frame}{Gaussian Elimination Algorithm}
    \begin{algorithmic}[1]
        \Function{GE}{$n, (a_{ij})$}
            \For{$k = 1 \textbf{ to } n-1$}
                \For{$i = k + 1 \textbf{ to } n$}
                    \Let{$z$}{$a_{ik}/a_{kk}$}
                    \Let{$a_{ik}$}{$0$}
                    \For{$j = k + 1 \textbf{ to } n$}
                        \Let{$a_{ij}$}{$a_{ij}-za_{kj}$}
                    \EndFor
                \EndFor
            \EndFor
            \vspace{-5mm}
            \State \Return{$(a_{ij})$}
        \EndFunction
    \end{algorithmic}
\end{frame}


\begin{frame}{Gaussian Elimination and Factorizations}
    \bi
        \item Recall: row operations are equivalent to left multiplication of coefficient matrix by elementary matrices \pause
        \bi
            \item Example: scale row 2 of $A \in \mathbb{R}^{3\text{x}3}$ by 4 $$EA,\quad E = \begin{bmatrix}1&0&0\\0&4&0\\0&0&1\end{bmatrix}$$
        \ei \pause
        \item Define G.E. recursively in terms of multiplication by these matrices: $$A_{k+1} = E_kA_k$$ \vspace{-5mm} \pause
        \item Note that algorithm terminates after $s = T_n$ operations with $A_{n\text{x}n}$ \pause
        \item Define $U = A_{s+1}$ and observe that $$U = E_s\cdots E_1A \quad \text{or} \quad A = E^{-1}_1\cdots E^{-1}_sU = LU$$ \vspace{-5mm} \pause
        \item So, Gaussian elimination constructs an $LU$ factorization (and hence is functionally equivalent to the $LU$ factorization algorithm!)
    \ei
\end{frame}


\begin{frame}{Gaussian Elimination: Pivoting Methods}
    Pivoting Methods \pause
    \bi
        \item Problem: Gaussian elimination breaks down (spectacular examples pg. 167--169) \pause
        \item Solution: incorporate row interchange operations via permutation matrix (we call this pivoting) \pause
        \item Types of pivoting:
        \bi
            \item Partial/complete: pertains to the amount of the matrix examined to find where to pivot \pause
            \bi
                \item Partial: analyze one row
                \item Complete: analyze entire submatrix
            \ei \pause
            \item Unscaled/scaled: pertains to how the potential pivot locations are compared \pause
            \bi
                \item Unscaled: compare absolute value of entries directly
                \item Scaled: compare ratios of pivot entries to maximum entries in magnitude in each row
            \ei
        \ei
    \ei
\end{frame}


\begin{frame}{Pivoting Algorithms}
    \begin{algorithmic}[1]
        \Function{GE\_PP}{$n, (a_{ij}), (p_{i})$}
            \For{$k = 1 \textbf{ to } n-1$}
                \For{$i = k + 1 \textbf{ to } n$}
                    \Let{$z$}{$a_{p_ik}/a_{p_kk}$}
                    \Let{$a_{p_ik}$}{$0$}
                    \For{$j = k + 1 \textbf{ to } n$}
                        \Let{$a_{p_ij}$}{$a_{p_ij}-za_{p_kj}$}
                    \EndFor
                \EndFor
            \EndFor
            \vspace{-5mm}
            \State \Return{$(a_{ij})$}
        \EndFunction
    \end{algorithmic}
\end{frame}


\begin{frame}{Pivoting Examples}
    \begin{columns}
    \column{.35\textwidth}
        \only<1-2>{
            $A_1=\begin{bmatrix}
            1 & 3 & 6 \\
            2 & 1 & 1 \\
            1 & 3 & 3
            \end{bmatrix}$
        }
        \only<3-5>{
            $A_2=\begin{bmatrix}
            0 & 2.5 & 5.5 \\
            1 & 0.5 & 0.5 \\
            0 & 2.5 & 2.5
            \end{bmatrix}$
        }
        \only<6-7>{
            $A_3=\begin{bmatrix}
            0 & 1 & 2.2 \\
            1 & 0.5 & $-0.6$ \\
            0 & 0 & $-3$
            \end{bmatrix}$
        }
        \only<8-8>{
            $A_4=\begin{bmatrix}
            0 & 1 & 2.2 \\
            1 & 0.5 & $-0.6$ \\
            0 & 0 & 1
            \end{bmatrix}$
        }
        \vspace{5mm}
        \only<1-1>{
            $p = \begin{bmatrix}1,&2,&3\end{bmatrix}$
        }
        \only<2-8>{
            $p = \begin{bmatrix}2,&1,&3\end{bmatrix}$
        }

    \column{.65\textwidth}
        \begin{algorithmic}[1]
            \Function{GE\_PP}{$n, (a_{ij}), (p_{i})$}
                \For{$k = 1 \textbf{ to } n-1$}
                    \For{$i = k + 1 \textbf{ to } n$}
                        \Let{$z$}{$a_{p_ik}/a_{p_kk}$}
                        \Let{$a_{p_ik}$}{$0$}
                        \For{$j = k + 1 \textbf{ to } n$}
                            \Let{$a_{p_ij}$}{$a_{p_ij}-za_{p_kj}$}
                        \EndFor
                    \EndFor
                \EndFor
                \vspace{-5mm}
                \State \Return{$(a_{ij})$}
            \EndFunction
        \end{algorithmic}
    \end{columns}

    \vspace{8mm}

    \only<1-3>{$$\max(\abs{1},\abs{2},\abs{1}) = 2$$}
    \only<2-3>{$$R_1 \leftarrow R_1 - \frac{1}{2}R_2$$}
    \only<2-3>{$$R_3 \leftarrow R_3 - \frac{1}{2}R_2$$}
    \only<2-3>{$$R_2 \leftarrow \frac{1}{2}R_2$$}
    \only<4-6>{$$\max(\abs{2.5},\abs{2.5}) = 2.5$$}
    \only<5-6>{$$R_3 \leftarrow R_3 - R_2$$}
    \only<5-6>{$$R_1 \leftarrow \frac{2}{5}R_1$$}
    \only<7-8>{$$R_3 \leftarrow -\frac{1}{3}R_3$$}
\end{frame}


\end{document}
