\documentclass[9pt, serif]{beamer}

% This file is a solution template for:

% - Talk at a conference/colloquium.
% - Talk length is about 20min.
% - Style is ornate.

\usepackage{psfrag}
\usepackage{graphics}
\usepackage{latexsym}
\usepackage{amsfonts}
\usepackage{amsmath}  
\usepackage{bm}
\usepackage{epstopdf}
\usepackage[noend]{algpseudocode}
\usepackage[english]{babel}
\usepackage[latin1]{inputenc}
\usepackage{helvet}
\usepackage[T1]{fontenc}

\newcommand{\bi}{\begin{itemize}}
\newcommand{\ei}{\end{itemize}}
\newcommand{\be}{\begin{enumerate}}
\newcommand{\ee}{\end{enumerate}}
\newcommand{\beq}{\begin{equation}}
\newcommand{\eeq}{\end{equation}}
\newcommand{\beqs}{\begin{equation*}}
\newcommand{\eeqs}{\end{equation*}}
\newcommand*\Let[2]{\State #1 $\gets$ #2}

\definecolor{royalblue}{rgb}{.2,.6,1}%
\definecolor{mygrey}{rgb}{.64,.64,.64}%
\definecolor{mywhite}{rgb}{1,1,1}%
\definecolor{mypurple}{rgb}{.8,.6,1}%

\definecolor{red}{rgb}{1,0,0}
\definecolor{skyblue}{rgb}{0.1960, 0.6000, 0.8000}


\mode<presentation>
{
	% THEME CHOICES
%	\usetheme{CambridgeUS}
	\usetheme{Warsaw}
%	\usetheme{Rochester}
%	\usetheme{Bergen}
%	\usetheme{Berlin}
%	\usetheme{Copenhagen}
%	\usetheme{Boadilla}
%	\usetheme{PaloAlto}
	% SET BACKGROUND COLORS
	\setbeamercolor{background canvas}{bg=white}
	\setbeamerfont{frametitle}{size=\large}
	\setbeamercolor{frame}{bg=mygold}
}


\title[Secant method]{Horner's Method and Finding Roots of Polynomials}
\author[]{Nate DeMaagd, Kurt O'Hearn}
\institute[Grand Valley State University]{MTH 499-02}
\date{February 18, 2013}


% If you have a file called "university-logo-filename.xxx", where xxx
% is a graphic format that can be processed by latex or pdflatex,
% resp., then you can add a logo as follows:

\pgfdeclareimage[height=1cm, width=1cm]{GVlogo}{GVSULOGOBLACK}
 \logo{\pgfuseimage{GVlogo}}

% to mask background in logo image
% \pgfdeclaremask{logobacking}{image name }
%


% Delete this, if you do not want the table of contents to pop up at
% the beginning of each subsection:
%\AtBeginSubsection[]
%{
 % \begin{frame}<beamer>{Outline}
 %   \tableofcontents[currentsection,currentsubsection]
 % \end{frame}
%}


% If you wish to uncover everything in a step-wise fashion, uncomment
% the following command:

%\beamerdefaultoverlayspecification{<+->}

\begin{document}

\begin{frame}
	\titlepage
\end{frame}

%\begin{frame}{Outline}
 % \tableofcontents
  % You might wish to add the option [pausesections]
%\end{frame}


\begin{frame}{\hspace{50mm}Introduction}
    \LARGE{Our outline for today is\ldots }
    \pause
    \bi
        \item Development of Horner's Algorithm
        \pause
        \item Pseudocode
        \pause
        \item Applications and Connections to Root Finding
        \pause
        \item Related Ideas: Julia Sets and Fractals
    \ei
\end{frame}


\begin{frame}{Development of Horner's Algorithm}
    Horner's Algorithm (Polynomials)
    \bi
\item Consider a polynomial $p$ with degree $n$ and some point $x_0$.
    We form the degree $n-1$ polynomial $q$ as: 
        \begin{align*} 
        		q(x)=\dfrac{p(x)-p(x_0)}{x-x_0}.
	\end{align*}
	\pause
\item Rearranging, we have
	\beq \label{EQ:1}
		p(x)=(x-x_0)q(x)+p(x_0).
	\eeq
    \pause
\item Expanding $p$ and $q$:
	\begin{align*}
		p(x)&=a_0+a_1x+a_2x^2+\dots+a_{n-2}x^{n-2}+a_{n-1}x^{n-1}+a_n x^n \\
		q(x)&=b_0+b_1x+b_2x^2+\dots+b_{n-2}x^{n-2}+b_{n-1}x^{n-1}.
	\end{align*}
    \ei
\end{frame}



\begin{frame}{Development of Horner's Algorithm Cont.}
\bi
    \item Equating like coefficients in \eqref{EQ:1}, we have
    \pause
    \begin{align*}
    	\text{constant}:\quad& p(x_0)=a_0+b_0x_0\\
	    x:\quad& b_0=a_1+b_1x_0\\
    	x^2:\quad& b_1=a_2+b_2x_0\\
	    \phantom{x:}\quad& ~\vdots\\
    	x^{n-1}:\quad& b_{n-2}=a_{n-1}+b_{n-1}x_0\\
	    x^n:\quad& b_{n-1}=a_n.
    \end{align*}
    \pause
    \item Notice: repeated substitution starting with $b_{n-1}$ to find all $b_j$ and $p(x_0)$
    \pause
    \item Question: why should we care?
\ei
\end{frame}


\begin{frame}{Horner's Algorithm: Pseudocode}
    % go check out minted and listing packages
    \begin{algorithmic}[1]
        \Function{Horner}{$n, [a_0, \ldots, a_n], x_0$}
	 \Let{$b_{n-1}$}{$a_n$}
            \For{$k = 0 \textbf{ to } n-1 \textbf{ step } -1$}
                \Let{$b_{k-1}$}{$a_k+b_k x_0$}
            \EndFor
            \State \Return{$[b_{-1}, \ldots, b_{n-1}]$}
        \EndFunction
    \end{algorithmic}
    \vspace{5mm}
    where \\
    \bi
	\item[] $n$: degree of $p$
	\item[] $a_i$: coefficients of $i^{\text{th}}$ degree term in $p$ (degree $n$)
	\item[] $b_j$: coefficients of $j^{\text{th}}$ degree term in $q$ (degree $n-1$)
	\item[] $x_0$: evaluation point
    \ei
\end{frame}


\begin{frame}{Significance of Horner's Method}
    Applications
    \pause
    \bi
        \item Function evaluation and Deflation \pause
        \item Taylor Expansions \pause
        \item Newton's method
    \ei
\end{frame}


\begin{frame}{Application: Function Evaluation and Deflation}
	\bi
		\item Recall: Given $p$ (degree $n$) and some $x_0$, Horner's algorithm finds all coefficients $b_j$
		of $q$ (degree $n-1$) and $p(x_0)$
		\pause
		\item Example: Deflate \vspace{-2mm} $$p(x) = x^3 - 2x^2 - 5x + 6, \quad x_0 = -1$$
		\pause
		\vspace{-5mm} \item Use synthetic division table:
		\pause
	\ei
		\begin{table}
			\centering
			\begin{tabular}{r | r r r r r r}
				& $a_n$ & $a_{n-1}$ & $a_{n-2}$ & $\cdots$ & $a_1$ & $a_0$ \\
				$x_0$ &   & $a_n x_0$ & $b_{n-1} x_0$ & $\cdots$ & $b_1 x_0$ & $b_0 x_0$ \\
				\hline
	   			& $a_n$ & $a_{n-1}+a_n x_0$ & $a_{n-2}+b_{n-1} x_0$ & $\cdots$ & $a_1+b_1 x_0$ & $a_0+b_0 x_0$
			\end{tabular}
	\end{table}
	\pause
	\begin{table}
		\centering
		\begin{tabular}{r | r r r r}
	   		& 1 & -2 & -5 & 6 \\
			-1 &    & \visible<7->{-1} & \visible<9->3 & \visible<11->2 \\
			\hline
	   		& \visible<6->1 & \visible<8->{-3} & \visible<10->{-2} & \visible<12->8
		\end{tabular}
	\end{table}
	\pause
	\bi
		\item<13-> Thus, $p(x) = (x+1)(x^2-3x-2)+8$
	\ei
\end{frame}


%TAYLOR EXPANSION
\begin{frame}{Application: Taylor Expansion}
\bi
\item Consider: $n$ term Taylor expansion of $p(x)$ about $x = x_0$: \pause
	\beqs
		p(x) = p(x_0) + \frac{p'(x_0)}{1!}(x-x_0) + \ldots + \frac{p^{(n)}(x_0)}{n!}(x-x_0)^n
	\eeqs \pause
\vspace{-5mm} \item Relabeling coefficients, we have \pause
	\beqs
		p(x) = c_n(x-x_0)^n+c_{n-1}(x-x_0)^{n-1}+\dots+c_0
	\eeqs \pause
\vspace{-5mm} \item Now, observe that repeatedly deflating $p$ at $x_0$ yields \pause
	\begin{align*}
		p(x)&=(x-x_0)q_1(x)+p(x_0)\\
		\visible<3->{&=(x-x_0)[(x-x_0)q_2(x)+q_1(x_0)]+p(x_0)\\}
		\visible<4->{&~~~~~~~~~~~~~~~~\vdots\\}
		\visible<5->{&=\sum_{i=1}^n \left[(x-x_0)^iq_i(x_0)\right]+p(x_0).}
	\end{align*}
	\pause
\vspace{-5mm} \item<6-> Thus, the remainder terms from Horner's method are the coefficients we seek!
\ei
\end{frame}


%\begin{frame}
%\bi
%\item Base step: Horner's algorithm gives $c_0=p(x_0)$.
%	\pause
%\item Second coefficient, $c_1$, found applying the algorithm again:
%	\begin{align*}
%		q(x)=c_n(x-x_0)^{n-1}+c_{n-1}(x-x_0)^{n-2}+\dots+c_1.
%	\end{align*}
%	\pause
%\item Repeat until all coefficients found.
%\ei
%\end{frame}


%TAYLOR EXAMPLE
\begin{frame}{Application: Taylor Expansion Cont.}
\bi
\item Example: Consider $$f(x)=x^3-2x^2-5x+6.$$
	Find Taylor expansion centered at $x=2$.
	\pause

\begin{columns}
\column{.33\textwidth}
\begin{table}\onslide<2->
\caption{Finding $c_0$}
		\centering
	\begin{tabular}{r | r r r r}
	   & 1 & -2 & -5 & 6 \\
	2 &    &  2 & 0 & -10 \\
	\hline
	   & 1 & 0 & -5 & \bf{-4}
	\end{tabular}
	\end{table}
\column{.33\textwidth}
\begin{table}\onslide<3->
\caption{Finding $c_1$}
		\centering
	\begin{tabular}{r | r r r}
	   & 1 & 0 & -5  \\
	2 &    &  2 & 4 \\
	\hline
	   & 1 & 2 &\bf{-1}
	\end{tabular}
	\end{table}
\column{.33\textwidth}
\begin{table}\onslide<4->
\caption{Finding $c_2$}
		\centering
	\begin{tabular}{r | r r}
	   & 1 & 2 \\
	2 &    &  2 \\
	\hline
	   & 1 & \bf{4} 
	\end{tabular}
	\end{table}
\end{columns}
    	\pause
	\item<5-> So, the Taylor expansion centered at $x=2$ is
	\visible<6->{
		$$f(x)=(x-2)^3+4(x-2)^2-(x-2)-4.$$
	}
	\ei
\end{frame}


\begin{frame}{Application: Newton's Method}
	\bi
		\item Recall: Newton's method is defined as $$x_{n+1} = x_n - \frac{f(x_n)}{f'(x_n)}$$
		\pause
		\item If $f$ is a polynomial, use H.A. to evaluate $f(x_n)$ and $f'(x_n)$
		\pause
		\item How: H.A. can evaluate functions and Taylor series!
		\pause
		\item Code: evaluate both points simultaneously (pg. 115)
	\ei
\end{frame}


\begin{frame}{Extension: Julia Sets and Fractals}
	\bi
		\item Consider $p$ and its roots $\xi_i$
		\pause
		\item Basin of attraction of $\xi_i$: set of all points which converge to $\xi_i$ when iterated (say, by N.M.)
		\pause
		\item Question: what about the points that don't converge to any root?
		\pause
		\item Answer: form Julia set
		\pause
		\item Demo!
	\ei
\end{frame}

\end{document}
