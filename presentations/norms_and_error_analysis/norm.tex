\documentclass[9pt, serif]{beamer}

\newlength{\wideitemsep}
\setlength{\wideitemsep}{\itemsep}
%\addtolength{\wideitemsep}{9pt}
\let\olditem\item
\renewcommand{\item}{\setlength{\itemsep}{\wideitemsep}\olditem}

% This file is a solution template for:

% - Talk at a conference/colloquium.
% - Talk length is about 20min.
% - Style is ornate.

\usepackage[noend]{algpseudocode}
\usepackage{psfrag}
\usepackage{graphics}
\usepackage{marvosym}
\usepackage{latexsym}
\usepackage{amsfonts}
\usepackage{amsmath}  
\usepackage{bm}
\usepackage{epstopdf}
\usepackage{tikz}
\usetikzlibrary{arrows,matrix,positioning}

\newcommand{\bi}{\begin{itemize}}
\newcommand{\be}{\begin{enumerate}}
\newcommand{\ei}{\end{itemize}}
\newcommand{\ee}{\end{enumerate}}
\newcommand*\Let[2]{\State #1 $\gets$ #2}
\newcommand{\abs}[1]{\lvert#1\rvert}
\newcommand{\norm}[1]{\lVert#1\rVert}

\newcommand{\bfr}[1]{\begin{frame}{\hspace{50mm}#1}}
\newcommand{\efr}{\end{frame}}

\definecolor{royalblue}{rgb}{.2,.6,1}%
\definecolor{mygrey}{rgb}{.64,.64,.64}%
\definecolor{mywhite}{rgb}{1,1,1}%
\definecolor{mypurple}{rgb}{.8,.6,1}%

\definecolor{red}{rgb}{1,0,0}
\definecolor{skyblue}{rgb}{0.1960, 0.6000, 0.8000}


\mode<presentation>
{
% THEME CHOICES
%\usetheme{CambridgeUS}
\usetheme{Warsaw}
%\usetheme{Rochester}
%\usetheme{Bergen}
%\usetheme{Berlin}
%\usetheme{Copenhagen}
%\usetheme{Boadilla}
%\usetheme{PaloAlto}

% SET BACKGROUND COLORS

\setbeamercolor{background canvas}{bg=white}
\setbeamerfont{frametitle}{size=\large}
}

\setbeamercolor{frame}{bg=mygold}

\usepackage[english]{babel}

\usepackage[latin1]{inputenc}

\usepackage{helvet}
\usepackage[T1]{fontenc}
\title[Norms and Error Analysis]
{You Perturb It, You Pay For It According to $\mathcal{K}(A)$:\\Norms and Error Analysis}



\author[]
{Nate DeMaagd, Kurt O'Hearn}


\institute[Grand Valley State University]
{MTH 499-02}
  
\date{March 25, 2013}


% If you have a file called "university-logo-filename.xxx", where xxx
% is a graphic format that can be processed by latex or pdflatex,
% resp., then you can add a logo as follows:

\pgfdeclareimage[height=1cm, width=1cm]{GVlogo}{GVSULOGOBLACK}
 \logo{\pgfuseimage{GVlogo}}

% to mask background in logo image
% \pgfdeclaremask{logobacking}{image name }
%


% Delete this, if you do not want the table of contents to pop up at
% the beginning of each subsection:
%\AtBeginSubsection[]
%{
 % \begin{frame}<beamer>{Outline}
 %   \tableofcontents[currentsection,currentsubsection]
 % \end{frame}
%}


% If you wish to uncover everything in a step-wise fashion, uncomment
% the following command:

%\beamerdefaultoverlayspecification{<+->}

\begin{document}

\begin{frame}
    \titlepage
\end{frame}

\begin{frame}{Outline}
    \bi
        \item Norms: properties, types, examples \pause
        \item Theorems, results on norms \pause
        \item Relative/absolute error in perturbations, conditioning number
    \ei
\end{frame}


%Definition of Norm
\begin{frame}{Vector Norms}
    \pause
    \begin{definition}
        Let $V$ be a vector space.  A norm is a function $\norm{\cdot}: V \to \mathbb{R}^+$ that
        has the following properties:
        \bi
            \item $\norm{x}>0$ if $x\neq0,~x\in V$
            \item $\norm{\lambda x} = \abs{\lambda}~\norm{x}$ if $\lambda\in\mathbb{R},~x\in V$
            \item $\norm{x+y}\leq\norm{x} + \norm{y}$ for all $x,y \in V$ (triangle inequality)
        \ei
    \end{definition} \pause
    Other comments:
\bi
    \item Norm can be thought of as generalization of absolute value
    \item Interpretation: vector length in $\mathbb{R}^2$, $\mathbb{R}^3$
\ei
\end{frame}


%TYPES OF NORMS IN TABLE FORM
\begin{frame}{Types of Norms}
    \pause
    Vector Norms:
    \pause
    \bi
    \item $l_k$ norm
        \begin{align*}
            l_k = \norm{x}_k = \left(\sum\limits_{i=1}^n \abs{x_i}^k\right)^{1/k}~\text{where}~k\in\{1,2,\dots\}~\text{and}~x=(x_1,\dots,x_n)^T
        \end{align*}
    \pause

\item Some common instances of the $l_k$ norm:
\begin{table}[H]
        \centering
    \begin{tabular}{c | l}

     $k$ & $l_k$ \\
     \hline
     1 & $\norm{x}_1 = \sum\limits_{i=1}^n \abs{x_i}$ \\
     2 & $\norm{x}_2=\left(\sum\limits_{i=1}^n x_i^2\right)^{1/2}$ \\
    $\infty$ & $\norm{x}_\infty = \smash{\displaystyle\max_{1 \leq i \leq n}} \abs{x_i}$
    \end{tabular}
    \end{table}

\ei
\end{frame}


%VECTOR NORM EXAMPLES
\begin{frame}{Examples of Vector Norms}
    \bi
        \item Consider vectors in $\mathbb{R}^3$
        \begin{align*}
            x = (2,-1,3)^T~~~~~ y = (7,2,5)^T~~~~~ z = (-3,1,-9)^T
        \end{align*}

        \begin{table}[H]
            \centering
            \begin{tabular}{c || c c c}
                & $||\cdot||_1$ & $||\cdot||_2$ & $||\cdot||_\infty$ \\
                \hline
                \hline
                $x$ & \visible<2->{6} & \visible<3->{3.74} & \visible<4->{3} \\
                $y$ & \visible<5->{14} & \visible<5->{8.83} & \visible<5->{7} \\
                $z$ & \visible<5->{13} & \visible<5->{9.54} & \visible<5->{9} \\
            \end{tabular}
        \end{table}

        \vspace{4mm}

        \only<2-2>{\centering{$||x||_1 = 2 + 1 +3 = 6$}}
        \only<3-3>{\centering{$||x||_2 = \sqrt{2^2 + 1^2 + 3^2} = 3.74$}}
        \only<4-4>{\centering{$||x||_\infty = 3$}}
    \ei
\end{frame}


%MATRIX NORMS
\begin{frame}{Matrix Norms}
    \begin{definition}
        A matrix norm subordinate to a vector norm $\norm{\cdot}$ is defined as
        \begin{align*}
            \norm{A} = \sup\left\{\norm{Au} : u\in\mathbb{R}^n , \norm{u} = 1\right\},
        \end{align*}
        where $A$ is an $n\times n$ matrix.
    \end{definition}
    \pause
    \bi
        \item Two common matrix norms:
        \pause
        \bi
            \item Subordinate to vector norm $\norm{\cdot}_\infty$ is
            \begin{align*}
                \norm{A}_\infty = \smash{\displaystyle\max_{1 \leq i \leq n}} \sum\limits_{j=1}^n \abs{a_{ij}},
            \end{align*}
            which is the maximum absolute row sum of $A$
            \pause

            \item Subordinate to vector norm $\norm{\cdot}_2$ is
            \begin{align*}
                \norm{A}_2 &= \sup\limits_{\norm{x}_2=1} \left\{\norm{Ax}_2\right\}\\
                &= \sqrt{\rho(A^TA)},
            \end{align*}
            where $\rho(A^TA)$ is spectral radius of $A^TA$ (i.e., largest eigenvalue in absolute value of $A^TA$)
        \ei
    \ei
\end{frame}


%||A||_\infty EXAMPLE
\begin{frame}{Examples of Matrix Norms}
    \bi
        \item Consider matrix
        $A = \begin{bmatrix}
            1 & 1 \\
            2 & 1\\
        \end{bmatrix}$
        \pause
        \item$(A^TA) = \begin{bmatrix} 5 & 3 \\ 3 & 2 \end{bmatrix}$
        \pause
        \item For $A^TA$, eigenvalues are $\sigma^2_1 = 6.8541$ and $\sigma^2_2 = 0.1459$.
        \pause
        \item So, $\norm{A}_2 = \sqrt{6.8541} \approx 2.618$.
    \ei
\end{frame}


%||A||_2 EXAMPLE
\begin{frame}{Examples of Matrix Norms}
    \bi
        \item Consider the matrix
        $B = \begin{bmatrix}
            3 & 2 & 4 \\
            2 & 0 & 2 \\
            4 & 2 & 3 \\
        \end{bmatrix}$
        \pause
        \item $\norm{B}_\infty$ is simply the maximum absolute row sum, which is 9.
    \ei
\end{frame}
%%%%%%%%%%%%%%%%%%%%%%%%%%%%%%%%%%%%%%%


\begin{frame}{Theorems and Results on Norms}
    \pause
    \begin{theorem}[Subordinate Matrix Norm]
        If $\norm{\cdot}$ is any norm in $\mathbb{R}^n$, then the equation 
        $$\norm{A} = \sup_{\norm{u}=1}\left\{\norm{Au}:u\in\mathbb{R}^n\right\}$$ defines a norm on the linear space of all $n \times n$
        matrices.
    \end{theorem} \pause
    \begin{proof}
        Follows from properties of vector norms.  Questions?
    \end{proof} \pause
    \begin{block}{Other Results}
        \bi
            \item $\norm{Ax} \le \norm{A}\norm{x} \qquad \left(x\in\mathbb{R}^n\right)$
            \item $\norm{AB} \le \norm{A}\norm{B}$
            \item $\norm{I} = 1$
        \ei
    \end{block}
\end{frame}


\begin{frame}{Error and Condition Number}
    \pause
    \begin{block}{Results}
        \bi
            \item Let $Ax = b$ be a linear system with $A$ an $n \times n$ invertible matrix, and let $B$ be a perturbation of
            $A$ such that $B\tilde{x} = b$. Then the following holds: $$\frac{\norm{x-\tilde{x}}}{\norm{x}} \le \norm{I-BA}$$ \pause
            \vspace{-5mm}
            \item Let $Ax = b$ be a linear system with $A$ an $n \times n$ invertible matrix, and let $\tilde{b}$ be a perturbation of
            $b$ such that $A\tilde{x} = \tilde{b}$. Then the following holds: $$\norm{x-\tilde{x}} \le \norm{A^{-1}}\,\norm{b-\tilde{b}}$$ \pause
            \vspace{-5mm}
            \item Putting these together, we get 
            $$\frac{\norm{x-\tilde{x}}}{\norm{x}} \le \mathcal{K}(A)\frac{\norm{b-\tilde{b}}}{\norm{b}}$$ where 
            $\mathcal{K}(A) = \norm{A}\,\norm{A^{-1}}$ is called the condition number of $A$ \pause
            \item $\mathcal{K}(A)$ large, $A$ ill-conditioned, else $A$ well-conditioned \pause
            \item $\mathcal{K}(A)>0$
        \ei
    \end{block}
\end{frame}


\begin{frame}{Error and Condition Number Cont.}
    \pause
    \begin{definition}
        We define the error $e$ and residual $r$ for perturbed systems as follows: 
        $$e = x - \tilde{x} \quad \text{and} \quad r = b - \tilde{b} = Ax - A\tilde{x} = Ae$$
    \end{definition} \pause
    \begin{block}{Results}
        \bi
            \item For $Ax = b$, we have 
            $$\frac{1}{\mathcal{K}(A)}\frac{\norm{r}}{\norm{b}} \le \frac{\norm{e}}{\norm{x}} \le \mathcal{K}(A)\frac{\norm{r}}{\norm{b}}$$
            \pause
            \item Question: why useful? \pause
            \item Answer: $b$, $r$, $\mathcal{K}(A)$ known, so can estimate the relative error in $x$
        \ei
    \end{block}
\end{frame}


\begin{frame}{Error and Conditino Number Example}
    Consider $Ax = b$ perturbed to $A\tilde{x} = \tilde{b}$ where 
    $$A = \begin{bmatrix} 1000 & 999\\999 & 998\end{bmatrix}, \quad b = \begin{bmatrix} 1999\\1997\end{bmatrix} \quad A^{-1} = \begin{bmatrix} -998 & 999\\999 & -1000\end{bmatrix} \quad \tilde{b}=\begin{bmatrix} 1998.99\\1997.01\end{bmatrix}$$
    $$\norm{A}_\infty = \norm{A^{-1}}_\infty = 1999, \quad \mathcal{K}(A) \approx 3.996 \times 10^6$$
    \pause
    Compute $$\frac{\norm{x-\tilde{x}}_\infty}{\norm{x}_\infty} \le \mathcal{K}(A)\frac{\norm{b-\tilde{b}}_\infty}{\norm{b}_\infty}$$
\end{frame}

\end{document}
